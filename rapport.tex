%%%%%%%%%%%%%%%%%%%%%%%%%%%%%%%%%%%%%%%%%
%%%%%%%%%%%%%%%%%%%%%%%%%%%%%%%%%%%%%%%%%
% Thin Sectioned Essay
% LaTeX Template
% Version 1.0 (3/8/13)
%
% This template has been downloaded from:
% http://www.LaTeXTemplates.com
%
% Original Author:
% Nicolas Diaz (nsdiaz@uc.cl) with extensive modifications by:
% Vel (vel@latextemplates.com)
%
% License:
% CC BY-NC-SA 3.0 (http://creativecommons.org/licenses/by-nc-sa/3.0/)
%
%%%%%%%%%%%%%%%%%%%%%%%%%%%%%%%%%%%%%%%%%

%----------------------------------------------------------------------------------------
% PACKAGES AND OTHER DOCUMENT CONFIGURATIONS
%----------------------------------------------------------------------------------------

\documentclass[a4paper, 11pt]{article} % Font size (can be 10pt, 11pt or 12pt) and paper size (remove a4paper for US letter paper)

\usepackage[protrusion=true,expansion=true]{microtype} % Better typography
\usepackage[utf8]{inputenc}  
\usepackage{graphicx} % Required for including pictures
\usepackage{wrapfig} % Allows in-line images
\usepackage[a4paper, margin={3cm, 3cm}]{geometry}

\usepackage{mathenv}
\usepackage{amsmath,amsfonts,amssymb}
\usepackage{setspace}
\usepackage{bbm}
\usepackage{bm}
\usepackage{layout}
\usepackage[]{algorithm2e}

\usepackage{mathpazo} % Use the Palatino font
\usepackage[T1]{fontenc} % Required for accented characters
\linespread{1.05} % Change line spacing here, Palatino benefits from a slight increase by default

\usepackage{listings} % Pour ajouter du code
\usepackage{xcolor}
\lstset { %
    language=C++,
    backgroundcolor=\color{black!5}, % set backgroundcolor
    basicstyle=\footnotesize,% basic font setting
}

\makeatletter
\renewcommand\@biblabel[1]{\textbf{#1.}} % Change the square brackets for each bibliography item from '[1]' to '1.'
\renewcommand{\@listI}{\itemsep=0pt} % Reduce the space between items in the itemize and enumerate environments and the bibliography

\renewcommand{\maketitle}{ % Customize the title - do not edit title and author name here, see the TITLE block below
\begin{flushright} % Right align
{\LARGE\@title} % Increase the font size of the title

\vspace{30pt} % Some vertical space between the title and author name

{\large\@author} % Author name
\\\@date % Date

\vspace{20pt} % Some vertical space between the author block and abstract
\end{flushright}
}

%----------------------------------------------------------------------------------------
% TITLE
%----------------------------------------------------------------------------------------

\title{\textbf{Modélisation projet C9-1}\\ % Title
GOTIC} % Subtitle

\author{\textsc{Dufour Maxime, Jay Emmanuel} % Author
\\{\textit{ENSTA ParisTech}}} % Institution

\date{\today} % Date

%----------------------------------------------------------------------------------------

\begin{document}

\vspace{200pt}

\maketitle % Print the title section

\section{Approche par relaxation lagrangienne}

\subsection*{QUESTION 3}

\paragraph*{}
$F$ la fonction de coût peut être écrite sous cettte forme :
$$ F(n,f,l) = \sum_{w \in W} \sum_{i \in J} d_B(w,i)(f_i^w + l_i^w)
            + \sum_{i,j\; / \;j \ne i} d_L(i,j)(\sum_{w \in W} n_{i,j}^w) $$ 
La contrainte qu'on relache est :
$$ \sum_{w \in W} x_i^w >= 1 \qquad \forall i \in J $$
On en déduit que le pseudo-lagrangien qu'on doit maintenant minimiser est :
$$ L(\lambda,n,f,l,x) = F(n,f,l) + <\lambda,\sum_{w \in W} x^w - 1> $$
Et la fonction duale associée :
$$ d(\lambda) = \inf_{n,f,l,x} {L(\lambda,n,f,l,x) } $$ 
sous le reste des contraintes.

\paragraph*{}
On peut donc écrire le pseudo-lagrangien sous la forme : 
$$ L(\lambda,n,f,l,x) = \sum_{w \in W} (\sum_{i \in J} d_B(w,i)(f_i^w + l_i^w)
            + \sum_{i,j\; / \;j \ne i} d_L(i,j) n_{i,j}^w + \sum_{i \in J} \lambda_i(x^w_i-\frac{1}{|W|}) ) $$
Ou l'on identifie le lagrangien par technicien : 
$$ l^w(n,f,l,x,\lambda) = \sum_{i \in J} d_B(w,i)(f_i^w + l_i^w)
            + \sum_{i,j\; / \;j \ne i} d_L(i,j) n_{i,j}^w + \sum_{i \in J} \lambda_i(x^w_i-\frac{1}{|W|}) $$
Qu'on doit minimiser pour tout lambda, étant tous indépendants, comme les contraintes restantes ne lient pas les différents techniciens.

\subsection*{QUESTION 4}

\paragraph*{}
On doit donc résoudre le problème suivant :
$$ \max_{\lambda} \sum_{w \in W} \min_{n^w,f^w,l^w,x^w} \{ l^w(n^w,f^w,l^w,x^w,\lambda)  \;\; / \; Contraintes \; non \; relachées \; \} $$
Et donc on en déduit qu'un sur-gradient de cette fonction au point $\lambda$ est, avec $X_0 = (n_0, f_0, l_0, x_0)$ optimaux :
$$ g(X_0) = \sum_{w \in W} x_0^w - 1 $$


\subsection*{QUESTION 5}

\paragraph*{}
% https://en.wikibooks.org/wiki/LaTeX/Algorithms
\begin{algorithm}[H]
 \KwData{Toutes les données précédentes}
 \KwResult{$\lambda$ et $X$ solutions du problème principal}
 initialisation : $k = 0$ et $\lambda^0 \in R_{+}^I $ \;
 \While {La fonction duale n'est pas stabilisée : $|d(\lambda^{k-1})-d(\lambda^{k-2})| \geq \epsilon$}{
  $d(\lambda^k) = 0$ \;
  \For{Chaque technicien $w$} 
    {On résout avec Cplex : $ d^w(\lambda^k) = \min l^w(n^w,f^w,l^w,x^w,\lambda^k)$
    sous les contraintes non relachées.\;
    $d(\lambda^k) += d^w(\lambda^k)$\;}
  $\lambda^{k+1} := (\lambda^k + \rho^k(\sum_{w \in W} x^w - 1))^{+}$ \;
  $k := k+1$ \;
 }
 \caption{Relaxation lagrangienne}
\end{algorithm}




\end{document}

